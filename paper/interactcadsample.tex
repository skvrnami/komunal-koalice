% interactcadsample.tex
% v1.04 - May 2023

\documentclass[]{interact}

\usepackage{epstopdf}% To incorporate .eps illustrations using PDFLaTeX, etc.
\usepackage{subfigure}% Support for small, `sub' figures and tables
%\usepackage[nolists,tablesfirst]{endfloat}% To `separate' figures and tables from text if required

\usepackage{natbib}% Citation support using natbib.sty
\bibpunct[, ]{(}{)}{;}{a}{}{,}% Citation support using natbib.sty
\renewcommand\bibfont{\fontsize{10}{12}\selectfont}% Bibliography support using natbib.sty

\theoremstyle{plain}% Theorem-like structures provided by amsthm.sty
\newtheorem{theorem}{Theorem}[section]
\newtheorem{lemma}[theorem]{Lemma}
\newtheorem{corollary}[theorem]{Corollary}
\newtheorem{proposition}[theorem]{Proposition}

\theoremstyle{definition}
\newtheorem{definition}[theorem]{Definition}
\newtheorem{example}[theorem]{Example}

\theoremstyle{remark}
\newtheorem{remark}{Remark}
\newtheorem{notation}{Notation}

\begin{document}

\articletype{ARTICLE TEMPLATE}

\title{Taylor \& Francis \LaTeX\ template for authors (\textsf{Interact} layout + Chicago author-date reference style)}

\author{
\name{A.~N. Author\textsuperscript{a}\thanks{CONTACT A.~N. Author. Email: latex.helpdesk@tandf.co.uk} and John Smith\textsuperscript{b}}
\affil{\textsuperscript{a}Taylor \& Francis, 4 Park Square, Milton Park, Abingdon, UK; \textsuperscript{b}Institut f\"{u}r Informatik, Albert-Ludwigs-Universit\"{a}t, Freiburg, Germany}
}

\maketitle

\begin{abstract}
This template is for authors who are preparing a manuscript for a Taylor \& Francis journal using the \LaTeX\ document preparation system and the \texttt{interact} class file, which is available via selected journals' home pages on the Taylor \& Francis website.
\end{abstract}

\begin{keywords}
Sections; lists; figures; tables; mathematics; fonts; references; appendices
\end{keywords}


\section{Introduction}

In order to assist authors in the process of preparing a manuscript for a journal, the Taylor \& Francis `\textsf{Interact}' layout style has been implemented as a \LaTeXe\ class file based on the \texttt{article} document class. A \textsc{Bib}\TeX\ bibliography style file and a sample bibliography are also provided in order to assist with the formatting of your references.

Commands that differ from or are provided in addition to standard \LaTeXe\ are described in this document, which is \emph{not} a substitute for a \LaTeXe\ tutorial.

The \texttt{interactcadsample.tex} file can be used as a template for a manuscript by cutting, pasting, inserting and deleting text as appropriate, using the preamble and the \LaTeX\ environments provided (e.g.\ \verb"\begin{abstract}", \verb"\begin{keywords}").


\subsection{The \textsf{Interact} class file}\label{class}

The \texttt{interact} class file preserves the standard \LaTeXe\ interface such that any document that can be produced using \texttt{article.cls} can also be produced with minimal alteration using the \texttt{interact} class file as described in this document.

If your article is accepted for publication it will be typeset as the journal requires in Minion Pro and/or Myriad Pro. Since most authors will not have these fonts installed, the page make-up is liable to alter slightly with the change of font. Also, the \texttt{interact} class file produces only single-column format, which is preferred for peer review and will be converted to two-column format by the typesetter if necessary during preparation of the proofs. Please therefore do not try to match the typeset format exactly, but use the standard \LaTeX\ fonts instead and ignore details such as slightly long lines of text or figures/tables not appearing in exact synchronization with their citations in the text: these details will be dealt with by the typesetter. Similarly, it is unnecessary to spend time addressing warnings in the log file -- if your .tex file compiles to produce a PDF document that correctly shows how you wish your paper to appear, such warnings will not prevent your source files being imported into the typesetter's program.


\subsection{Submission of manuscripts prepared using \emph{\LaTeX}}

Manuscripts for possible publication should be submitted to the Editors for review as directed in the journal's Instructions for Authors, and in accordance with any technical instructions provided in the journal's ScholarOne Manuscripts or Editorial Manager site. Your \LaTeX\ source file(s), the class file and any graphics files will be required in addition to the final PDF version when final, revised versions of accepted manuscripts are submitted.

Please ensure that any author-defined macros used in your article are gathered together in the preamble of your .tex file, i.e.\ before the \verb"\begin{document}" command. Note that if serious problems are encountered in the coding of a document (missing author-defined macros, for example), the typesetter may resort to rekeying it.


\section{Using the \texttt{interact} class file}

For convenience, simply copy the \texttt{interact.cls} file into the same directory as your manuscript files (you do not need to install it in your \TeX\ distribution). In order to use the \texttt{interact} document class, replace the command \verb"\documentclass{article}" at the beginning of your document with the command \verb"\documentclass{interact}".

The following document-class options should \emph{not} be used with the \texttt{interact} class file:
\begin{itemize}
  \item \texttt{10pt}, \texttt{11pt}, \texttt{12pt} -- unavailable;
  \item \texttt{oneside}, \texttt{twoside} -- not necessary, \texttt{oneside} is the default;
  \item \texttt{leqno}, \texttt{titlepage} -- should not be used;
  \item \texttt{twocolumn} -- should not be used (see Subsection~\ref{class});
  \item \texttt{onecolumn} -- not necessary as it is the default style.
\end{itemize}
To prepare a manuscript for a journal that is printed in A4 (two column) format, use the \verb"largeformat" document-class option provided by \texttt{interact.cls}; otherwise the class file produces pages sized for B5 (single column) format by default. The \texttt{geometry} package should not be used to make any further adjustments to the page dimensions.

%If your manuscript has supplementary content you can prepare this using the \verb"suppldata" document-class option, which will suppress the `article history' date. This option must \emph{not} be used on any primary content.


\section{Additional features of the \texttt{interact} class file}

\subsection{Title, authors' names and affiliations, abstracts and article types}

The title should be generated at the beginning of your article using the \verb"\maketitle" command.
In the final version the author name(s) and affiliation(s) must be followed immediately by \verb"\maketitle" as shown below in order for them to be displayed in your PDF document.
To prepare an anonymous version for double-blind peer review, you can put the \verb"\maketitle" between the \verb"\title" and the \verb"\author" in order to hide the author name(s) and affiliation(s) temporarily.
Next you should include the abstract if your article has one, enclosed within an \texttt{abstract} environment.
The \verb"\articletype" command is also provided as an \emph{optional} element which should only be included if your article actually needs it.
For example, the titles for this document begin as follows:
\begin{verbatim}
\articletype{ARTICLE TEMPLATE}

\title{Taylor \& Francis \LaTeX\ template for authors (\textsf{Interact}
layout + Chicago author-date reference style)}

\author{
\name{A.~N. Author\textsuperscript{a}\thanks{CONTACT A.~N. Author.
Email: latex.helpdesk@tandf.co.uk} and John Smith\textsuperscript{b}}
\affil{\textsuperscript{a}Taylor \& Francis, 4 Park Square, Milton
Park, Abingdon, UK; \textsuperscript{b}Institut f\"{u}r Informatik,
Albert-Ludwigs-Universit\"{a}t, Freiburg, Germany} }

\maketitle

\begin{abstract}
This template is for authors who are preparing a manuscript for a
Taylor \& Francis journal using the \LaTeX\ document preparation system
and the \texttt{interact} class file, which is available via selected
journals' home pages on the Taylor \& Francis website.
\end{abstract}
\end{verbatim}

An additional abstract in another language (preceded by a translation of the article title) may be included within the \verb"abstract" environment if required.

A graphical abstract may also be included if required. Within the \verb"abstract" environment you can include the code
\begin{verbatim}
\\\resizebox{25pc}{!}{\includegraphics{abstract.eps}}
\end{verbatim}
where the graphical abstract is to appear, where \verb"abstract.eps" is the name of the file containing the graphic (note that \verb"25pc" is the recommended maximum width, expressed in pica, for the graphical abstract in your manuscript).


\subsection{Abbreviations}

A list of abbreviations may be included if required, enclosed within an \texttt{abbreviations} environment, i.e.\ \verb"\begin{abbreviations}"\ldots\verb"\end{abbreviations}", immediately following the \verb"abstract" environment.


\subsection{Keywords}

A list of keywords may be included if required, enclosed within a \texttt{keywords} environment, i.e.\ \verb"\begin{keywords}"\ldots\verb"\end{keywords}". Additional keywords in other languages (preceded by a translation of the word `keywords') may also be included within the \verb"keywords" environment if required.


\subsection{Subject classification codes}

AMS, JEL or PACS classification codes may be included if required. The \texttt{interact} class file provides an \texttt{amscode} environment, i.e.\ \verb"\begin{amscode}"\ldots\verb"\end{amscode}", a \texttt{jelcode} environment, i.e.\ \verb"\begin{jelcode}"\ldots\verb"\end{jelcode}", and a \texttt{pacscode} environment, i.e.\ \verb"\begin{pacscode}"\ldots\verb"\end{pacscode}" to assist with this.


\subsection{Additional footnotes to the title or authors' names}

The \verb"\thanks" command may be used to create additional footnotes to the title or authors' names if required. Footnote symbols for this purpose should be used in the order
$^\ast$~(coded as \verb"$^\ast$"), $\dagger$~(\verb"$\dagger$"), $\ddagger$~(\verb"$\ddagger$"), $\S$~(\verb"$\S$"), $\P$~(\verb"$\P$"), $\|$~(\verb"$\|$"),
$\dagger\dagger$~(\verb"$\dagger\dagger$"), $\ddagger\ddagger$~(\verb"$\ddagger\ddagger$"), $\S\S$~(\verb"$\S\S$"), $\P\P$~(\verb"$\P\P$").

Note that any \verb"footnote"s to the main text will automatically be assigned the superscript symbols 1, 2, 3, etc. by the class file.\footnote{If preferred, the \texttt{endnotes} package may be used to set the notes at the end of your text, before the bibliography. The symbols will be changed to match the style of the journal if necessary by the typesetter.}


\section{Some guidelines for using the standard features of \LaTeX}

\subsection{Sections}

The \textsf{Interact} layout style allows for five levels of section heading, all of which are provided in the \texttt{interact} class file using the standard \LaTeX\ commands \verb"\section", \verb"\subsection", \verb"\subsubsection", \verb"\paragraph" and \verb"\subparagraph". Numbering will be automatically generated for all these headings by default.


\subsection{Lists}

Numbered lists are produced using the \texttt{enumerate} environment, which will number each list item with arabic numerals by default. For example,
\begin{enumerate}
  \item first item
  \item second item
  \item third item
\end{enumerate}
was produced by
\begin{verbatim}
\begin{enumerate}
  \item first item
  \item second item
  \item third item
\end{enumerate}
\end{verbatim}
Alternative numbering styles can be achieved by inserting an optional argument in square brackets to each \verb"item", e.g.\ \verb"\item[(i)] first item"\, to create a list numbered with roman numerals at level one.

Bulleted lists are produced using the \texttt{itemize} environment. For example,
\begin{itemize}
  \item First bulleted item
  \item Second bulleted item
  \item Third bulleted item
\end{itemize}
was produced by
\begin{verbatim}
\begin{itemize}
  \item First bulleted item
  \item Second bulleted item
  \item Third bulleted item
\end{itemize}
\end{verbatim}


\subsection{Figures}

The \texttt{interact} class file will deal with positioning your figures in the same way as standard \LaTeX. It should not normally be necessary to use the optional \texttt{[htb]} location specifiers of the \texttt{figure} environment in your manuscript; you may, however, find the \verb"[p]" placement option or the \verb"endfloat" package useful if a journal insists on the need to separate figures from the text.

Figure captions appear below the figures themselves, therefore the \verb"\caption" command should appear after the body of the figure. For example, Figure~\ref{sample-figure} with caption and sub-captions is produced using the following commands:
\begin{verbatim}
\begin{figure}
\centering
\subfigure[An example of an individual figure sub-caption.]{%
\resizebox*{5cm}{!}{\includegraphics{graph1.eps}}}\hspace{5pt}
\subfigure[A slightly shorter sub-caption.]{%
\resizebox*{5cm}{!}{\includegraphics{graph2.eps}}}
\caption{Example of a two-part figure with individual sub-captions
 showing that captions are flush left and justified if greater
 than one line of text.} \label{sample-figure}
\end{figure}
\end{verbatim}
\begin{figure}
\centering
\subfigure[An example of an individual figure sub-caption.]{%
\resizebox*{5cm}{!}{\includegraphics{graph1.eps}}}\hspace{5pt}
\subfigure[A slightly shorter sub-caption.]{%
\resizebox*{5cm}{!}{\includegraphics{graph2.eps}}}
\caption{Example of a two-part figure with individual sub-captions
 showing that captions are flush left and justified if greater
 than one line of text.} \label{sample-figure}
\end{figure}

To ensure that figures are correctly numbered automatically, the \verb"\label" command should be included just after the \verb"\caption" command, or in its argument.

The \verb"\subfigure" command requires \verb"subfigure.sty", which is called in the preamble of the \texttt{interacttfssample.tex} file (to allow your choice of an alternative package if preferred) and included in the \textsf{Interact} \LaTeX\ bundle for convenience. Please supply any additional figure macros used with your article in the preamble of your .tex file.

The source files of any figures will be required when the final, revised version of a manuscript is submitted. Authors should ensure that these are suitable (in terms of lettering size, etc.) for the reductions they envisage.

The \texttt{epstopdf} package can be used to incorporate encapsulated PostScript (.eps) illustrations when using PDF\LaTeX, etc. Please provide the original .eps source files rather than the generated PDF images of those illustrations for production purposes.


\subsection{Tables}

The \texttt{interact} class file will deal with positioning your tables in the same way as standard \LaTeX. It should not normally be necessary to use the optional \texttt{[htb]} location specifiers of the \texttt{table} environment in your manuscript; you may, however, find the \verb"[p]" placement option or the \verb"endfloat" package useful if a journal insists on the need to separate tables from the text.

The \texttt{tabular} environment can be used as shown to create tables with single horizontal rules at the head, foot and elsewhere as appropriate. The captions appear above the tables in the \textsf{Interact} style, therefore the \verb"\tbl" command should be used before the body of the table. For example, Table~\ref{sample-table} is produced using the following commands:
\begin{table}
\tbl{Example of a table showing that its caption is as wide as
 the table itself and justified.}
{\begin{tabular}{lcccccc} \toprule
 & \multicolumn{2}{l}{Type} \\ \cmidrule{2-7}
 Class & One & Two & Three & Four & Five & Six \\ \midrule
 Alpha\textsuperscript{a} & A1 & A2 & A3 & A4 & A5 & A6 \\
 Beta & B2 & B2 & B3 & B4 & B5 & B6 \\
 Gamma & C2 & C2 & C3 & C4 & C5 & C6 \\ \bottomrule
\end{tabular}}
\tabnote{\textsuperscript{a}This footnote shows how to include
 footnotes to a table if required.}
\label{sample-table}
\end{table}
\begin{verbatim}
\begin{table}
\tbl{Example of a table showing that its caption is as wide as
 the table itself and justified.}
{\begin{tabular}{lcccccc} \toprule
 & \multicolumn{2}{l}{Type} \\ \cmidrule{2-7}
 Class & One & Two & Three & Four & Five & Six \\ \midrule
 Alpha\textsuperscript{a} & A1 & A2 & A3 & A4 & A5 & A6 \\
 Beta & B2 & B2 & B3 & B4 & B5 & B6 \\
 Gamma & C2 & C2 & C3 & C4 & C5 & C6 \\ \bottomrule
\end{tabular}}
\tabnote{\textsuperscript{a}This footnote shows how to include
 footnotes to a table if required.}
\label{sample-table}
\end{table}
\end{verbatim}

To ensure that tables are correctly numbered automatically, the \verb"\label" command should be included just before \verb"\end{table}".

The \verb"\toprule", \verb"\midrule", \verb"\bottomrule" and \verb"\cmidrule" commands are those used by \verb"booktabs.sty", which is called by the \texttt{interact} class file and included in the \textsf{Interact} \LaTeX\ bundle for convenience. Tables produced using the standard commands of the \texttt{tabular} environment are also compatible with the \texttt{interact} class file.


\subsection{Landscape pages}

If a figure or table is too wide to fit the page it will need to be rotated, along with its caption, through 90$^{\circ}$ anticlockwise. Landscape figures and tables can be produced using the \verb"rotating" package, which is called by the \texttt{interact} class file. The following commands (for example) can be used to produce such pages.
\begin{verbatim}
\setcounter{figure}{1}
\begin{sidewaysfigure}
\centerline{\epsfbox{figname.eps}}
\caption{Example landscape figure caption.}
\label{landfig}
\end{sidewaysfigure}
\end{verbatim}
\begin{verbatim}
\setcounter{table}{1}
\begin{sidewaystable}
 \tbl{Example landscape table caption.}
  {\begin{tabular}{@{}llllcll}
    .
    .
    .
  \end{tabular}}\label{landtab}
\end{sidewaystable}
\end{verbatim}
Before any such float environment, use the \verb"\setcounter" command as above to fix the numbering of the caption (the value of the counter being the number given to the preceding figure or table). Subsequent captions will then be automatically renumbered accordingly. The \verb"\epsfbox" command requires \verb"epsfig.sty", which is called by the \texttt{interact} class file and is also included in the \textsf{Interact} \LaTeX\ bundle for convenience.

Please note that if the \verb"endfloat" package is used, one or both of the commands
\begin{verbatim}
\DeclareDelayedFloatFlavor{sidewaysfigure}{figure}
\DeclareDelayedFloatFlavor{sidewaystable}{table}
\end{verbatim}
will need to be included in the preamble of your .tex file, after the \verb"endfloat" package is loaded, in order to process any landscape figures and/or tables correctly.


\subsection{Theorem-like structures}

A predefined \verb"proof" environment is provided by the \texttt{amsthm} package (which is called by the \texttt{interact} class file), as follows:
\begin{proof}
More recent algorithms for solving the semidefinite programming relaxation are particularly efficient, because they explore the structure of the MAX-CUT problem.
\end{proof}
\noindent This was produced by simply typing:
\begin{verbatim}
\begin{proof}
More recent algorithms for solving the semidefinite programming
relaxation are particularly efficient, because they explore the
structure of the MAX-CUT problem.
\end{proof}
\end{verbatim}
Other theorem-like environments (theorem, definition, remark, etc.) need to be defined as required, e.g.\ using \verb"\newtheorem{theorem}{Theorem}" in the preamble of your .tex file (see the preamble of \verb"interactcadsample.tex" for more examples). You can define the numbering scheme for these structures however suits your article best. Please note that the format of the text in these environments may be changed if necessary to match the style of individual journals by the typesetter during preparation of the proofs.


\subsection{Mathematics}

\subsubsection{Displayed mathematics}

The \texttt{interact} class file will set displayed mathematical formulas centred on the page without equation numbers if you use the \texttt{displaymath} environment or the equivalent \verb"\[...\]" construction. For example, the equation
\[
 \hat{\theta}_{w_i} = \hat{\theta}(s(t,\mathcal{U}_{w_i}))
\]
was typeset using the commands
\begin{verbatim}
\[
 \hat{\theta}_{w_i} = \hat{\theta}(s(t,\mathcal{U}_{w_i}))
\]
\end{verbatim}

For those of your equations that you wish to be automatically numbered sequentially throughout the text for future reference, use the \texttt{equation} environment, e.g.
\begin{equation}
 \hat{\theta}_{w_i} = \hat{\theta}(s(t,\mathcal{U}_{w_i}))
\end{equation}
was typeset using the commands
\begin{verbatim}
\begin{equation}
 \hat{\theta}_{w_i} = \hat{\theta}(s(t,\mathcal{U}_{w_i}))
\end{equation}
\end{verbatim}

Part numbers for sets of equations may be generated using the \texttt{subequations} environment, e.g.
\begin{subequations} \label{subeqnexample}
\begin{equation}
     \varepsilon \rho w_{tt}(s,t) = N[w_{s}(s,t),w_{st}(s,t)]_{s},
     \label{subeqnparta}
\end{equation}
\begin{equation}
     w_{tt}(1,t)+N[w_{s}(1,t),w_{st}(1,t)] = 0,
     \label{subeqnpartb}
\end{equation}
\end{subequations}
which was typeset using the commands
\begin{verbatim}
\begin{subequations} \label{subeqnexample}
\begin{equation}
     \varepsilon \rho w_{tt}(s,t) = N[w_{s}(s,t),w_{st}(s,t)]_{s},
     \label{subeqnparta}
\end{equation}
\begin{equation}
     w_{tt}(1,t)+N[w_{s}(1,t),w_{st}(1,t)] = 0,   \label{subeqnpartb}
\end{equation}
\end{subequations}
\end{verbatim}
This is made possible by the \texttt{amsmath} package, which is called by the class file. If you put a \verb"\label" just after the \verb"\begin{subequations}" command, references can be made to the collection of equations, i.e.\ `(\ref{subeqnexample})' in the example above. Or, as the example also shows, you can label and refer to each equation individually -- i.e.\ `(\ref{subeqnparta})' and `(\ref{subeqnpartb})'.

Displayed mathematics should be given end-of-line punctuation appropriate to the running text sentence of which it forms a part, if required.

\subsubsection{Math fonts}

\paragraph{Superscripts and subscripts}
Superscripts and subscripts will automatically come out in the correct size in a math environment (i.e.\ enclosed within \verb"\(...\)" or \verb"$...$" commands in running text, or within \verb"\[...\]" or the \texttt{equation} environment for displayed equations). Sub/superscripts that are physical variables should be italic, whereas those that are labels should be roman (e.g.\ $C_p$, $T_\mathrm{eff}$). If the subscripts or superscripts need to be other than italic, they must be coded individually.

\paragraph{Upright Greek characters and the upright partial derivative sign}
Upright lowercase Greek characters can be obtained by inserting the letter `u' in the control code for the character, e.g.\ \verb"\umu" and \verb"\upi" produce $\umu$ (used, for example, in the symbol for the unit microns -- $\umu\mathrm{m}$) and $\upi$ (the ratio of the circumference of a circle to its diameter). Similarly, the control code for the upright partial derivative $\upartial$ is \verb"\upartial". Bold lowercase as well as uppercase Greek characters can be obtained by \verb"{\bm \gamma}", for example, which gives ${\bm \gamma}$, and \verb"{\bm \Gamma}", which gives ${\bm \Gamma}$.


\section*{Acknowledgement(s)}

An unnumbered section, e.g.\ \verb"\section*{Acknowledgements}", may be used for thanks, etc.\ if required and included \emph{in the non-anonymous version} before any Notes or References.


\section*{Disclosure statement}

An unnumbered section, e.g.\ \verb"\section*{Disclosure statement}", may be used to declare any potential conflict of interest and included \emph{in the non-anonymous version} before any Notes or References, after any Acknowledgements and before any Funding information.


\section*{Funding}

An unnumbered section, e.g.\ \verb"\section*{Funding}", may be used for grant details, etc.\ if required and included \emph{in the non-anonymous version} before any Notes or References.


\section*{Notes on contributor(s)}

An unnumbered section, e.g.\ \verb"\section*{Notes on contributors}", may be included \emph{in the non-anonymous version} if required. A photograph may be added if requested.


\section*{Nomenclature/Notation}

An unnumbered section, e.g.\ \verb"\section*{Nomenclature}" (or \verb"\section*{Notation}"), may be included if required, before any Notes or References.


\section*{Notes}

An unnumbered `Notes' section may be included before the References (if using the \verb"endnotes" package, use the command \verb"\theendnotes" where the notes are to appear, instead of creating a \verb"\section*").


\section{References}

\subsection{References cited in the text}

References should be cited in Chicago author-date style, e.g.\ `\citep{Alb05,Gre08,Sch87}' or `\ldots~see Smith (1985,~75)'. If there are three authors, list them all in every citation, e.g.\ `\citep{Nowacki20}'. For more than three authors, cite the first author's name followed by et al. For two or more sources by the same author(s) in the same year, use lower-case letters (a, b, c, ...) with the year to order the entries in the References list and use these letters with the year in the in-text citations, e.g.\ `\citep{FogEHPD04,FogJEE04}'. If two or more authors have the same surname, use their initials with the surnames, e.g.\ `(C.~Doershuk 2010; J.~Doershuk 2009)'. If the first author's names and the years of publication are identical for several references, include enough co-author names to eliminate ambiguity, e.g.\ `(Schonen, Baker, et~al. 2009; Schonen, Brooks, et~al. 2009)'. For further details on this reference style, see the Instructions for Authors on the Taylor \& Francis website.

Each bibliographic entry has a key, which is assigned by the author and is used to refer to that entry in the text. In this document, the key \verb"Fow89" in the citation form \verb"\citep{Fow89}" produces `\citep{Fow89}', and the keys \verb"{Bro86,Bro02,Roh08}" in the citation form \verb"\citep{Bro86,Bro02,Prinzie09}" produce `\citep{Bro86,Bro02,Prinzie09}'. The appropriate citation style for different situations can be obtained, for example, by \verb"\citet{Draper1987}" for `\citet{Draper1987}', \verb"\citealt{Lev05}" for `\citealt{Lev05}', or \verb"\citealp{Mor08}" for `\citealp{Mor08}'. Citation of the year alone may be produced by \verb"\citeyear{Cho08}", i.e.\ `\citeyear{Cho08}', or \verb"\citeyearpar{ChoGul08}", i.e.\ `\citeyearpar{ChoGul08}', or of the author(s) alone by \verb"\citeauthor{Wittgenstein2009}", i.e.\ `\citeauthor{Wittgenstein2009}'. Optional notes may be included at the beginning and/or end of a citation by the use of square brackets, e.g.\ \verb"\citep[see][275]{Ell1968}" produces `\citep[see][275]{Ell1968}'; \verb"\citep[e.g.][]{Wau50}" produces `\citep[e.g.][]{Wau50}'; \verb"\citet[chap.~2]{Str00}" produces `\citet[chap.~2]{Str00}'. A `plain' \verb"\cite" command will produce the same result as a \verb"\citet", i.e.\ \verb"\cite{Wei02}" will produce `\cite{Wei02}'.


\subsection{The list of references}

References should be listed at the end of the main text in alphabetical order by authors' surnames, then chronologically (earliest first).
If references have the same author(s), editor(s), etc., arrange by year of publication, with undated works at the end.
A single-author entry precedes a multi-author entry that begins with the same name.
If the reference list contains two or more items by the same author(s) in the same year, add a, b, etc. and list them alphabetically by title.
Successive entries by two or more authors when only the first author is the same are alphabetized by co-authors' surnames.
If a reference has more than ten named authors, list only the first seven, followed by `et~al.'.
If a reference has no author or editor, order by title; if a date of publication is impossible to find, use `n.d.' in its place.

The following list shows some sample references prepared in the Taylor \& Francis Chicago author-date style.

\begin{thebibliography}{}

\bibitem[Albiston(2005)]{Alb05}
Albiston, Catherine~R. 2005. ``Bargaining in the Shadow of Social Institutions:
Competing Discourses and Social Change in the Workplace Mobilization of Civil
Rights.'' \emph{Law and Society Review} 39 (1): 11--47.
https://doi.org/{10.1111/j.0023-9216.2005.00076.x}.

\bibitem[Blair(1977)]{Blair1977}
Blair, Walter. 1977. ``Americanized Comic Braggarts.'' \emph{Critical Inquiry}
4 (2): 331--349. https://doi.org/{10.1086/447940}.

\bibitem[Brooks and McLennan(2002)]{Bro02}
Brooks, Daniel~R., and Deborah~A. McLennan. 2002. \emph{The Nature of
	Diversity: An Evolutionary Voyage of Discovery}. Chicago: University of
Chicago Press.

\bibitem[Brooks and Wiley(1986)]{Bro86}
Brooks, Daniel~R., and E.~O. Wiley. 1986. \emph{Evolution as Entropy}. 2nd ed.
Chicago: University of Chicago Press.

\bibitem[Bunn, Bennett, and Burke(2018)]{Bunn2018}
Bunn, Matthew, Anna Bennett, and Penny~Jane Burke. 2018. ``Temporalities of
Trust and Betrayal: Teaching and Learning in the Neoliberal University.''
Paper presented at the Society for Research in Higher Education Annual
Research Conference, Newport, UK, December 2018.

\bibitem[Choi(2008)]{Cho08}
Choi, Mihwa. 2008. ``Contesting \emph{Imaginaires} in Death Rituals during the
Northern {S}ong Dynasty.'' PhD diss., University of Chicago.
{http://pi.lib.uchicago.edu/1001/cat/bib/6820707}.

\bibitem[Choi and Gulati(2008)]{ChoGul08}
Choi, Stephen~J., and G.~Mitu Gulati. 2008. ``Bias in Judicial Citations: A
Window into the Behavior of Judges?'' \emph{Journal of Legal Studies} 37 (1):
87--129. https://doi.org/{10.2139/ssrn.913663}.

\bibitem[Draper(1987)]{Draper1987}
Draper, Joan~E. 1987. ``Paris by the Lake: Sources of {B}urnham's Plan of
{C}hicago.'' In \emph{Chicago Architecture, 1872--1922: Birth of a
	Metropolis},  edited by John Zukowsky, 107--119. Munich: Prestel-Verlag in
association with the Art Institute of Chicago.

\bibitem[Edelman(1992)]{Edelman1992}
Edelman, Lauren~B. 1992. ``Legal Ambiguity and Symbolic Structures:
Organizational Mediation of Civil Rights Law.'' \emph{American Journal of
	Sociology} 97 (6): 1531--1576. https://doi.org/{10.1086/229939}.

\bibitem[Edelman, Fuller, and Mara-Drita(2001)]{EdelmanFullerMara2001}
Edelman, Lauren~B., Sally~Riggs Fuller, and Iona Mara-Drita. 2001. ``Diversity
Rhetoric and the Managerialization of the Law.'' \emph{American Journal of
	Sociology} 106 (6): 1589--1641. https://doi.org/{10.1086/321303}.

\bibitem[Edelman, Uggen, and Erlanger(1999)]{EdelmanUggenErlanger1999}
Edelman, Lauren~B., Christopher Uggen, and Howard~S. Erlanger. 1999. ``The
Endogeneity of Legal Regulation: Grievance Procedures as Rational Myth.''
\emph{American Journal of Sociology} 105 (2): 406--454.
https://doi.org/{10.1086/210316}.

\bibitem[Ellet(1968)]{Ell1968}
Ellet, Elizabeth F.~L. 1968. ``By Rail and Stage to Galena.'' In \emph{Prairie
	State: Impressions of {I}llinois, 1673--1967, by Travelers and Other
	Observers},  edited by Paul~M. Angle, 271--279. Chicago: University of
Chicago Press.

\bibitem[Fogel(2004a)]{FogEHPD04}
Fogel, Robert~William. 2004a. \emph{The Escape from Hunger and
	Premature Death, 1700--2100: {E}urope, {A}merica, and the {T}hird {W}orld}.
New York: Cambridge University Press.

\bibitem[Fogel(2004b)]{FogJEE04}
Fogel, Robert~William. 2004b. ``Technophysio Evolution and the
Measurement of Economic Growth.'' \emph{Journal of Evolutionary Economics} 14
(2): 217--221. https://doi.org/{10.1007/s00191-004-0188-x}.

\bibitem[Fowler(1989)]{Fow89}
Fowler, Melvin~L. 1989. \emph{The {C}ahokia Atlas: A Historical Atlas of
	{C}ahokia Archaeology}. Studies in Illinois Archaeology~6. Springfield:
Illinois Historic Preservation Agency.

\bibitem[Gould(1984)]{Gould1984}
Gould, Glenn. 1984. ``Streisand as {S}chwarzkopf.'' In \emph{The {G}lenn
	{G}ould Reader},  edited by Tim Page, 308--311. New York: Vintage.

\bibitem[Greenberg(2008)]{Gre08}
Greenberg, Joel, ed. 2008. \emph{Of Prairie, Woods, and Water: Two Centuries of
	{C}hicago Nature Writing}. Chicago: University of Chicago Press.

\bibitem[Harrington(1987)]{Harrington1987}
Harrington, Elaine. 1987. ``International Influences on {H}enry {H}obson
{R}ichardson's {G}lessner {H}ouse.'' In \emph{Chicago Architecture,
	1872--1922: Birth of a Metropolis},  edited by John Zukowsky, 189--207.
Munich: Prestel-Verlag in association with the Art Institute of Chicago.

\bibitem[Harris and Zucker(2015)]{HarrisZucker2015}
Harris, Beth, and Steven Zucker. 2015. ``Haussmann the Demolisher and the
Creation of Modern Paris.'' Smarthistory.
{https://smarthistory.org/haussmann-the-demolisher-and-the-creation-of-modern-paris}.

\bibitem[Heinrich(2008)]{Heinrich2008}
Heinrich, Larissa. 2008. \emph{The Afterlife of Images: Translating the
	Pathological Body between {C}hina and the {W}est}. Durham, NC: Duke
University Press.

\bibitem[Jacobs, Thomas, and Lang(1997)]{Jacobs1997}
Jacobs, Sue-Ellen, Wesley Thomas, and Sabine Lang, eds. 1997. \emph{Two-Spirit
	People: {N}ative {A}merican Gender Identity, Sexuality, and Spirituality}.
Urbana: University of Illinois Press.

\bibitem[Kelly(2010)]{Kelly2010}
Kelly, John~D. 2010. ``Seeing Red: {M}ao Fetishism, {P}ax {A}mericana, and the
Moral Economy of War.'' In \emph{Anthropology and Global Counterinsurgency},
edited by John~D. Kelly, Beatrice Jauregui, Sean~T. Mitchell, and Jeremy
Walton, 67--83. Chicago: University of Chicago Press.

\bibitem[Kogan(1974)]{Kogan1974}
Kogan, Herman. 1974. \emph{The First Century: The {C}hicago {B}ar
	{A}ssociation, 1874--1974}. Chicago: Rand McNally.

\bibitem[Kogan and Wendt(1958)]{KoganWendt1958}
Kogan, Herman, and Lloyd Wendt. 1958. \emph{Chicago: A Pictorial History}. New
York: Dutton.

\bibitem[Levitt and Dubner(2005)]{Lev05}
Levitt, Steven~D., and Stephen~J. Dubner. 2005. \emph{Freakonomics: A Rogue
	Economist Explores the Hidden Side of Everything}. New York: William Morrow.

\bibitem[Mnookin and Kornhauser(1979)]{Mno79}
Mnookin, Robert, and Lewis Kornhauser. 1979. ``Bargaining in the Shadow of the
Law: The Case of Divorce.'' \emph{Yale Law Journal} 88 (5): 950--997.
https://doi.org/{10.2307/795824}.

\bibitem[Morasse, Guderley, and Dodson(2008)]{Mor08}
Morasse, S{\'e}bastien, Helga Guderley, and Julian~J. Dodson. 2008. ``Paternal
Reproductive Strategy Influences Metabolic Capacities and Muscle Development
of {A}tlantic Salmon (\emph{Salmo salar} L.) Embryos.'' \emph{Physiological
	and Biochemical Zoology} 81 (4): 402--413. https://doi.org/{10.1086/589012}.

\bibitem[Nowacki, Creech, and Parks(2020)]{Nowacki20}
Nowacki, J.~S., D.~Creech, and M.~Parks. 2020. ``Political Climate, Voter
Suppression Policies and Federal Sentencing Outcomes.'' \emph{British Journal
	of Criminology} 60 (5): 1115--1180. https://doi.org/{10.1093/bjc/azaa017}.

\bibitem[Pelikan(1971)]{Pelikan1971}
Pelikan, Jaroslav. 1971. \emph{The Emergence of the Catholic Tradition
	(100--600)}. Vol.~1 of \emph{The {C}hristian Tradition: A History of the
	Development of Doctrine}. Chicago: University of Chicago Press.

\bibitem[Piaget(1995)]{Piaget1995}
Piaget, Jean. 1995. ``Logical Operators and Social Life.'' Translated by Wolfe
Mays. In \emph{Sociological Studies},  edited by Leslie Smith, 134--157.
Hove, UK: Routledge.

\bibitem[Pollan(2006)]{Pollan2006}
Pollan, Michael. 2006. \emph{The Omnivore's Dilemma: A Natural History of Four
	Meals}. New York: Penguin.

\bibitem[Prinzie et~al.(2009)]{Prinzie09}
Prinzie, Peter, Geert Jan J.~M. Stams, Maja Dekovi{\'c}, Albert H.~A.
Reijntjes, and Jay Belsky. 2009. ``The Relations between Parents' Big Five
Personality Factors and Parenting: A Meta-analytic Review.'' \emph{Journal of
	Personality and Social Psychology} 97 (2): 351--362.
https://doi.org/{10.1037/a0015823}.

\bibitem[Schuman and Scott(1987)]{Sch87}
Schuman, Howard, and Jacqueline Scott. 1987. ``Problems in the Use of Survey
Questions to Measure Public Opinion.'' \emph{Science} 236 (4804): 957--959.
https://doi.org/{10.1126/science.236.4804.957}.

\bibitem[Schuman and Scott(1989)]{Sch89}
Schuman, Howard, and Jacqueline Scott. 1989. ``Generations and Collective
Memories.'' \emph{American Sociological Review} 54 (3): 359--381.
https://doi.org/{10.2307/2095611}.

\bibitem[Soltes(1999)]{Sol99}
Soltes, Ori~Z., ed. 1999. \emph{Georgia: Art and Civilization through the
	Ages}. London: Philip Wilson.

\bibitem[Stearns(1864)]{Stearns1864}
Stearns, A.~A. 1864. ``Armory {S}quare {H}ospital Nursing Diary.'' MS B 372.
History of Medicine Division. National Library of Medicine, Bethesda, MD.

\bibitem[Strunk and White(2000)]{Str00}
Strunk, William, Jr., and E.~B. White. 2000. \emph{The Elements of Style}. 4th
ed. New York: Allyn and Bacon.

\bibitem[Van~Wagenen and Simpson(1973)]{VanWagenen1973}
Van~Wagenen, Gertrude, and Miriam~E. Simpson. 1973. \emph{Postnatal Development
	of the Ovary in \emph{Homo sapiens} and \emph{Macaca mulatta} and Induction
	of Ovulation in the Macaque}. New Haven, CT: Yale University Press.

\bibitem[Wagmiller et~al.(2018)]{Wagmiller18}
Wagmiller, R.~L., K.~LaParo, D.~Schrider, Wildfeuer R, M.~Ford, and S.~M.
Hitchcock. 2018. ``Attitudes towards Interracial Marriage in a Diverse and
Unequal Metropolis: A Lost Letter Study.'' Poster presented at the American
Sociological Association 113th Annual Meeting, Philadelphia, PA, August
2018.

\bibitem[Wauchope(1950)]{Wau50}
Wauchope, Robert. 1950. \emph{A Tentative Sequence of Pre-Classic Ceramics in
	Middle {A}merica}. Vol.~1 of \emph{Middle {A}merican Research Records}. New
Orleans, LA: Tulane University.

\bibitem[Weber(2010)]{Weber2020}
Weber, Max. 2010. \emph{Politik als Beruf \emph{[Politics as vocation]}}. 11th
ed. Berlin: Duncker \& Humblot.

\bibitem[Weigel and Glazebrook(2002)]{Wei02}
Weigel, Detlef, and Jane Glazebrook. 2002. \emph{Arabidopsis: A Laboratory
	Manual}. Cold Spring Harbor, NY: Cold Spring Harbor Laboratory Press.

\bibitem[Wittgenstein(2009)]{Wittgenstein2009}
Wittgenstein, Ludwig. 2009. \emph{Philosophical Investigations}. Translated by
G.~E.~M. Anscombe, P.~M.~S. Hacker, and Joachim Schulte. 4th ed. Oxford:
Wiley--Blackwell.

\bibitem[Woodward(1977)]{Woodward1977}
Woodward, David. 1977. \emph{The All-{A}merican Map: Wax Engraving and Its
	Influence on Cartography}. Chicago: University of Chicago Press.

\bibitem[Woodward(1987)]{Woodward1987}
Woodward, David, ed. 1987. \emph{Art and Cartography: Six Historical Essays}.
Chicago: University of Chicago Press.

\bibitem[Woodward(1996)]{Woodward1996}
Woodward, David. 1996. \emph{Catalogue of Watermarks in {I}talian Printed Maps,
	ca.~1540--1600}. Chicago: University of Chicago Press.

\end{thebibliography}
\bigskip
\noindent This was produced by typing:
\begin{verbatim}
\begin{thebibliography}{}

\bibitem[Albiston(2005)]{Alb05}
Albiston, Catherine~R. 2005. ``Bargaining in the Shadow of Social Institutions:
Competing Discourses and Social Change in the Workplace Mobilization of Civil
Rights.'' \emph{Law and Society Review} 39 (1): 11--47.
https://doi.org/{10.1111/j.0023-9216.2005.00076.x}.

\bibitem[Blair(1977)]{Blair1977}
Blair, Walter. 1977. ``Americanized Comic Braggarts.'' \emph{Critical Inquiry}
4 (2): 331--349. https://doi.org/{10.1086/447940}.

\bibitem[Brooks and McLennan(2002)]{Bro02}
Brooks, Daniel~R., and Deborah~A. McLennan. 2002. \emph{The Nature of
	Diversity: An Evolutionary Voyage of Discovery}. Chicago: University of
Chicago Press.

\bibitem[Brooks and Wiley(1986)]{Bro86}
Brooks, Daniel~R., and E.~O. Wiley. 1986. \emph{Evolution as Entropy}. 2nd ed.
Chicago: University of Chicago Press.

\bibitem[Bunn, Bennett, and Burke(2018)]{Bunn2018}
Bunn, Matthew, Anna Bennett, and Penny~Jane Burke. 2018. ``Temporalities of
Trust and Betrayal: Teaching and Learning in the Neoliberal University.''
Paper presented at the Society for Research in Higher Education Annual
Research Conference, Newport, UK, December 2018.

\bibitem[Choi(2008)]{Cho08}
Choi, Mihwa. 2008. ``Contesting \emph{Imaginaires} in Death Rituals during the
Northern {S}ong Dynasty.'' PhD diss., University of Chicago.
{http://pi.lib.uchicago.edu/1001/cat/bib/6820707}.

\bibitem[Choi and Gulati(2008)]{ChoGul08}
Choi, Stephen~J., and G.~Mitu Gulati. 2008. ``Bias in Judicial Citations: A
Window into the Behavior of Judges?'' \emph{Journal of Legal Studies} 37 (1):
87--129. https://doi.org/{10.2139/ssrn.913663}.

\bibitem[Draper(1987)]{Draper1987}
Draper, Joan~E. 1987. ``Paris by the Lake: Sources of {B}urnham's Plan of
{C}hicago.'' In \emph{Chicago Architecture, 1872--1922: Birth of a
	Metropolis},  edited by John Zukowsky, 107--119. Munich: Prestel-Verlag in
association with the Art Institute of Chicago.

\bibitem[Edelman(1992)]{Edelman1992}
Edelman, Lauren~B. 1992. ``Legal Ambiguity and Symbolic Structures:
Organizational Mediation of Civil Rights Law.'' \emph{American Journal of
	Sociology} 97 (6): 1531--1576. https://doi.org/{10.1086/229939}.

\bibitem[Edelman, Fuller, and Mara-Drita(2001)]{EdelmanFullerMara2001}
Edelman, Lauren~B., Sally~Riggs Fuller, and Iona Mara-Drita. 2001. ``Diversity
Rhetoric and the Managerialization of the Law.'' \emph{American Journal of
	Sociology} 106 (6): 1589--1641. https://doi.org/{10.1086/321303}.

\bibitem[Edelman, Uggen, and Erlanger(1999)]{EdelmanUggenErlanger1999}
Edelman, Lauren~B., Christopher Uggen, and Howard~S. Erlanger. 1999. ``The
Endogeneity of Legal Regulation: Grievance Procedures as Rational Myth.''
\emph{American Journal of Sociology} 105 (2): 406--454.
https://doi.org/{10.1086/210316}.

\bibitem[Ellet(1968)]{Ell1968}
Ellet, Elizabeth F.~L. 1968. ``By Rail and Stage to Galena.'' In \emph{Prairie
	State: Impressions of {I}llinois, 1673--1967, by Travelers and Other
	Observers},  edited by Paul~M. Angle, 271--279. Chicago: University of
Chicago Press.

\bibitem[Fogel(2004a)]{FogEHPD04}
Fogel, Robert~William. 2004a. \emph{The Escape from Hunger and
	Premature Death, 1700--2100: {E}urope, {A}merica, and the {T}hird {W}orld}.
New York: Cambridge University Press.

\bibitem[Fogel(2004b)]{FogJEE04}
Fogel, Robert~William. 2004b. ``Technophysio Evolution and the
Measurement of Economic Growth.'' \emph{Journal of Evolutionary Economics} 14
(2): 217--221. https://doi.org/{10.1007/s00191-004-0188-x}.

\bibitem[Fowler(1989)]{Fow89}
Fowler, Melvin~L. 1989. \emph{The {C}ahokia Atlas: A Historical Atlas of
	{C}ahokia Archaeology}. Studies in Illinois Archaeology~6. Springfield:
Illinois Historic Preservation Agency.

\bibitem[Gould(1984)]{Gould1984}
Gould, Glenn. 1984. ``Streisand as {S}chwarzkopf.'' In \emph{The {G}lenn
	{G}ould Reader},  edited by Tim Page, 308--311. New York: Vintage.

\bibitem[Greenberg(2008)]{Gre08}
Greenberg, Joel, ed. 2008. \emph{Of Prairie, Woods, and Water: Two Centuries of
	{C}hicago Nature Writing}. Chicago: University of Chicago Press.

\bibitem[Harrington(1987)]{Harrington1987}
Harrington, Elaine. 1987. ``International Influences on {H}enry {H}obson
{R}ichardson's {G}lessner {H}ouse.'' In \emph{Chicago Architecture,
	1872--1922: Birth of a Metropolis},  edited by John Zukowsky, 189--207.
Munich: Prestel-Verlag in association with the Art Institute of Chicago.

\bibitem[Harris and Zucker(2015)]{HarrisZucker2015}
Harris, Beth, and Steven Zucker. 2015. ``Haussmann the Demolisher and the
Creation of Modern Paris.'' Smarthistory.
{https://smarthistory.org/haussmann-the-demolisher-and-the-creation-of-modern-paris}.

\bibitem[Heinrich(2008)]{Heinrich2008}
Heinrich, Larissa. 2008. \emph{The Afterlife of Images: Translating the
	Pathological Body between {C}hina and the {W}est}. Durham, NC: Duke
University Press.

\bibitem[Jacobs, Thomas, and Lang(1997)]{Jacobs1997}
Jacobs, Sue-Ellen, Wesley Thomas, and Sabine Lang, eds. 1997. \emph{Two-Spirit
	People: {N}ative {A}merican Gender Identity, Sexuality, and Spirituality}.
Urbana: University of Illinois Press.

\bibitem[Kelly(2010)]{Kelly2010}
Kelly, John~D. 2010. ``Seeing Red: {M}ao Fetishism, {P}ax {A}mericana, and the
Moral Economy of War.'' In \emph{Anthropology and Global Counterinsurgency},
edited by John~D. Kelly, Beatrice Jauregui, Sean~T. Mitchell, and Jeremy
Walton, 67--83. Chicago: University of Chicago Press.

\bibitem[Kogan(1974)]{Kogan1974}
Kogan, Herman. 1974. \emph{The First Century: The {C}hicago {B}ar
	{A}ssociation, 1874--1974}. Chicago: Rand McNally.

\bibitem[Kogan and Wendt(1958)]{KoganWendt1958}
Kogan, Herman, and Lloyd Wendt. 1958. \emph{Chicago: A Pictorial History}. New
York: Dutton.

\bibitem[Levitt and Dubner(2005)]{Lev05}
Levitt, Steven~D., and Stephen~J. Dubner. 2005. \emph{Freakonomics: A Rogue
	Economist Explores the Hidden Side of Everything}. New York: William Morrow.

\bibitem[Mnookin and Kornhauser(1979)]{Mno79}
Mnookin, Robert, and Lewis Kornhauser. 1979. ``Bargaining in the Shadow of the
Law: The Case of Divorce.'' \emph{Yale Law Journal} 88 (5): 950--997.
https://doi.org/{10.2307/795824}.

\bibitem[Morasse, Guderley, and Dodson(2008)]{Mor08}
Morasse, S{\'e}bastien, Helga Guderley, and Julian~J. Dodson. 2008. ``Paternal
Reproductive Strategy Influences Metabolic Capacities and Muscle Development
of {A}tlantic Salmon (\emph{Salmo salar} L.) Embryos.'' \emph{Physiological
	and Biochemical Zoology} 81 (4): 402--413. https://doi.org/{10.1086/589012}.

\bibitem[Nowacki, Creech, and Parks(2020)]{Nowacki20}
Nowacki, J.~S., D.~Creech, and M.~Parks. 2020. ``Political Climate, Voter
Suppression Policies and Federal Sentencing Outcomes.'' \emph{British Journal
	of Criminology} 60 (5): 1115--1180. https://doi.org/{10.1093/bjc/azaa017}.

\bibitem[Pelikan(1971)]{Pelikan1971}
Pelikan, Jaroslav. 1971. \emph{The Emergence of the Catholic Tradition
	(100--600)}. Vol.~1 of \emph{The {C}hristian Tradition: A History of the
	Development of Doctrine}. Chicago: University of Chicago Press.

\bibitem[Piaget(1995)]{Piaget1995}
Piaget, Jean. 1995. ``Logical Operators and Social Life.'' Translated by Wolfe
Mays. In \emph{Sociological Studies},  edited by Leslie Smith, 134--157.
Hove, UK: Routledge.

\bibitem[Pollan(2006)]{Pollan2006}
Pollan, Michael. 2006. \emph{The Omnivore's Dilemma: A Natural History of Four
	Meals}. New York: Penguin.

\bibitem[Prinzie et~al.(2009)]{Prinzie09}
Prinzie, Peter, Geert Jan J.~M. Stams, Maja Dekovi{\'c}, Albert H.~A.
Reijntjes, and Jay Belsky. 2009. ``The Relations between Parents' Big Five
Personality Factors and Parenting: A Meta-analytic Review.'' \emph{Journal of
	Personality and Social Psychology} 97 (2): 351--362.
https://doi.org/{10.1037/a0015823}.

\bibitem[Schuman and Scott(1987)]{Sch87}
Schuman, Howard, and Jacqueline Scott. 1987. ``Problems in the Use of Survey
Questions to Measure Public Opinion.'' \emph{Science} 236 (4804): 957--959.
https://doi.org/{10.1126/science.236.4804.957}.

\bibitem[Schuman and Scott(1989)]{Sch89}
Schuman, Howard, and Jacqueline Scott. 1989. ``Generations and Collective
Memories.'' \emph{American Sociological Review} 54 (3): 359--381.
https://doi.org/{10.2307/2095611}.

\bibitem[Soltes(1999)]{Sol99}
Soltes, Ori~Z., ed. 1999. \emph{Georgia: Art and Civilization through the
	Ages}. London: Philip Wilson.

\bibitem[Stearns(1864)]{Stearns1864}
Stearns, A.~A. 1864. ``Armory {S}quare {H}ospital Nursing Diary.'' MS B 372.
History of Medicine Division. National Library of Medicine, Bethesda, MD.

\bibitem[Strunk and White(2000)]{Str00}
Strunk, William, Jr., and E.~B. White. 2000. \emph{The Elements of Style}. 4th
ed. New York: Allyn and Bacon.

\bibitem[Van~Wagenen and Simpson(1973)]{VanWagenen1973}
Van~Wagenen, Gertrude, and Miriam~E. Simpson. 1973. \emph{Postnatal Development
	of the Ovary in \emph{Homo sapiens} and \emph{Macaca mulatta} and Induction
	of Ovulation in the Macaque}. New Haven, CT: Yale University Press.

\bibitem[Wagmiller et~al.(2018)]{Wagmiller18}
Wagmiller, R.~L., K.~LaParo, D.~Schrider, Wildfeuer R, M.~Ford, and S.~M.
Hitchcock. 2018. ``Attitudes towards Interracial Marriage in a Diverse and
Unequal Metropolis: A Lost Letter Study.'' Poster presented at the American
Sociological Association 113th Annual Meeting, Philadelphia, PA, August
2018.

\bibitem[Wauchope(1950)]{Wau50}
Wauchope, Robert. 1950. \emph{A Tentative Sequence of Pre-Classic Ceramics in
	Middle {A}merica}. Vol.~1 of \emph{Middle {A}merican Research Records}. New
Orleans, LA: Tulane University.

\bibitem[Weber(2010)]{Weber2020}
Weber, Max. 2010. \emph{Politik als Beruf \emph{[Politics as vocation]}}. 11th
ed. Berlin: Duncker \& Humblot.

\bibitem[Weigel and Glazebrook(2002)]{Wei02}
Weigel, Detlef, and Jane Glazebrook. 2002. \emph{Arabidopsis: A Laboratory
	Manual}. Cold Spring Harbor, NY: Cold Spring Harbor Laboratory Press.

\bibitem[Wittgenstein(2009)]{Wittgenstein2009}
Wittgenstein, Ludwig. 2009. \emph{Philosophical Investigations}. Translated by
G.~E.~M. Anscombe, P.~M.~S. Hacker, and Joachim Schulte. 4th ed. Oxford:
Wiley--Blackwell.

\bibitem[Woodward(1977)]{Woodward1977}
Woodward, David. 1977. \emph{The All-{A}merican Map: Wax Engraving and Its
	Influence on Cartography}. Chicago: University of Chicago Press.

\bibitem[Woodward(1987)]{Woodward1987}
Woodward, David, ed. 1987. \emph{Art and Cartography: Six Historical Essays}.
Chicago: University of Chicago Press.

\bibitem[Woodward(1996)]{Woodward1996}
Woodward, David. 1996. \emph{Catalogue of Watermarks in {I}talian Printed Maps,
	ca.~1540--1600}. Chicago: University of Chicago Press.

\end{thebibliography}
\end{verbatim}
\bigskip
\noindent Each entry takes the form:
\begin{verbatim}
\bibitem[authors' names(date of publication)]{key}
 Bibliography entry
\end{verbatim}
where `\texttt{authors' names}' is the list of names to appear where the \verb"bibitem" is cited in the text, and `\texttt{key}' is the tag that is to be used as an argument for the \verb"\cite{}" commands in the text of the article. `\texttt{Bibliography entry}' is the material that is to appear in the list of references, suitably formatted. The commands
\begin{verbatim}
\usepackage{natbib}
\bibpunct[, ]{(}{)}{;}{a}{}{,}
\renewcommand\bibfont{\fontsize{10}{12}\selectfont}
\end{verbatim}
need to be included in the preamble of your .tex file in order to generate the citations and bibliography as described above.

Instead of typing the bibliography by hand, you may prefer to create the list of references using a \textsc{Bib}\TeX\ database. The \texttt{tfcad.bst} file needs to be in your working folder or an appropriate directory, and the lines
\begin{verbatim}
\bibliographystyle{tfcad}
\bibliography{interactcadsample}
\end{verbatim}
included where the list of references is to appear, where \texttt{tfcad.bst} is the name of the \textsc{Bib}\TeX\ bibliography style file for Taylor \& Francis' Chicago author-date reference style and \texttt{interactcadsample.bib} is the bibliographic database included with the \textsf{Interact}-CAD \LaTeX\ bundle (to be replaced with the name of your own .bib file). \LaTeX/\textsc{Bib}\TeX\ will extract from your .bib file only those references that are cited in your .tex file and list them in the References section.

Please include a copy of your .bib file and/or the final generated .bbl file among your source files if your .tex file does not contain a reference list in a \texttt{thebibliography} environment.


\section{Appendices}

Any appendices should be placed after the list of references, beginning with the command \verb"\appendix" followed by the command \verb"\section" for each appendix title, e.g.
\begin{verbatim}
\appendix
\section{This is the title of the first appendix}
\section{This is the title of the second appendix}
\end{verbatim}
produces:\medskip

\noindent\textbf{Appendix A. This is the title of the first appendix}\medskip

\noindent\textbf{Appendix B. This is the title of the second appendix}\medskip

\noindent Subsections, equations, figures, tables, etc.\ within appendices will then be automatically numbered as appropriate. Some theorem-like environments may need to have their counters reset manually (e.g.\ if they are not numbered within sections in the main text). You can achieve this by using \verb"\numberwithin{remark}{section}" (for example) just after the \verb"\appendix" command.

Please note that if the \verb"endfloat" package is used on a document containing appendices, the \verb"\processdelayedfloats" command must be included immediately before the \verb"\appendix" command in order to ensure that the floats in the main body of the text are numbered as such.

%\processdelayedfloats %%% See above for an explanation of why this command might be needed.

\appendix

\section{Troubleshooting}

Authors may occasionally encounter problems with the preparation of a manuscript using \LaTeX. The appropriate action to take will depend on the nature of the problem:
\begin{enumerate}
\item[(i)] If the problem is with \LaTeX\ itself, rather than with the actual macros, please consult an appropriate \LaTeXe\ manual for initial advice. If the solution cannot be found, or if you suspect that the problem does lie with the macros, then please contact Taylor \& Francis for assistance (\texttt{latex.helpdesk@tandf.co.uk}).
\item[(ii)] Problems with page make-up (e.g.\ occasional overlong lines of text; figures or tables appearing out of order): please do not try to fix these using `hard' page make-up commands -- the typesetter will deal with such problems. (You may, if you wish, draw attention to particular problems when submitting the final version of your manuscript.)
\item[(iii)] If a required font is not available on your system, allow \TeX\ to substitute the font and specify which font is required in a covering letter accompanying your files.
\end{enumerate}


\section{Obtaining the template and class file}

\subsection{Via the Taylor \& Francis website}

This article template and the \texttt{interact} class file may be obtained via the `Instructions for Authors' pages of selected Taylor \& Francis journals.

Please note that the class file calls up the open-source \LaTeX\ packages booktabs.sty, epsfig.sty and rotating.sty, which will, for convenience, unpack with the downloaded template and class file. The template calls for natbib.sty and subfigure.sty, which are also supplied for convenience.


\subsection{Via e-mail}

This article template, the \texttt{interact} class file and the associated open-source \LaTeX\ packages are also available via e-mail. Requests should be addressed to \texttt{latex.helpdesk@tandf.co.uk}, clearly stating for which journal you require the template and class file.

\end{document}
